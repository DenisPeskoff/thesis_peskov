\section{Our Dataset in Context}

Addressing discourse phenomena is important for high-quality \abr{mt}.  
%
Apart from document-level coherence and cohesion, anaphoric pronoun translation has proven to be  an important testing ground for the ability of context-aware \nmt{} to model discourse. 
%
Anaphoric pronoun translation is the focus of several works in context-aware \nmt{}~\citep{bawden-etal-2018-evaluating,voita2018anaphora,stojanovski-fraser-2018-coreference,miculicich2018documentnmt,voita2019good,maruf2019selective}. 

\citet{bawden-etal-2018-evaluating} manually create such a contrastive challenge set for English$\rightarrow$French pronoun translation. 
%
ContraPro~\citep{mueller2018} follows this work, but creates the challenge set in an automatic way. 
%
We show that making small variations in ContraPro substantially changes the accuracy scores, precipitating our new dataset.


\citet{jwalapuram-etal-2019-evaluating} propose 
a model for pronoun translation evaluation 
trained on pairs of sentences
consisting of the reference and a system output with differing pronouns.
%
However, as \citet{guillou-hardmeier-2018-automatic} point out, this fails to take into account that often there is not a 1:1 correspondence between pronouns in different languages and that a system translation may be
correct
despite not containing the exact
pronoun in the reference, and incorrect even if containing the pronoun in the reference, because of differences in the translation of the referent.
%
Moreover, introducing a separate model which needs to be trained before evaluation adds an extra layer of complexity in the evaluation setup and makes interpretability more difficult. 
In contrast, templates can easily be used to pinpoint specific issues of an \nmt{} model.
%
Our templates follow previous work \citep{ribeiro-etal-2018-semantically,mccoy2019right,ribeiro2020beyond}
where similar tests are proposed for diagnosing \textsc{nlp} models.

