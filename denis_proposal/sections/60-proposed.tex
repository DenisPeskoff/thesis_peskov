\chapter{Proposed Work}

%intro
\label{ch:proposal}

%CUT by comment Our past work establishes that experts, Chapters~\ref{ch:contracat} and \ref{ch:expert} can solve tasks not possible by generalists, Chapters ~\ref{ch:unspecialized} and \ref{ch:hybrid}.
%
%Additionally, the work independently creates datasets for machine translation, Background Section~\ref{sec:mt} and question answering, Background Section~\ref{sec:qa}. 

We propose a new task where the gold standard is subjective and all-important, thereby requiring authoritative experts.  
%
Machine translation usually translates words literally; however, this does not necessarily apply in a cultural context.  
%
Certain named entities may be relevant in one culture but not another.  
%
One can find applicable named entity modulations by referencing WikiData, a human-interpretable and human-verified representation of Wikipedia.
%
We will want to investigate if this method generates better candidates than an embedding-based approach, such as word2vec.  
%
And a genuine evaluation of this approach requires specialized users, specifically German nationals that would understand the language and culture.  

\section{Using Cultural Experts for Translation}
\label{sec:propmod}
Chapter~\ref{ch:hybrid} proposes a method to evaluate machine translation models and in turn data.   
%
%MOVED TO earlier chapter %If we can establish that neural models are shallow in their understanding of a task, we should be able to establish that current auto-generated or crowd-sourced datasets are insufficient in quality.
%
The major limitation of the work is that the crowd-sourced workers generate less authentic sentences than the experts.  
%
How then can we generate data at scale, but with a level of reliability? 

Adaptation is such a task that combines our past work in Question Answering, with our proposed work in Machine Translation and is a good, difficult test-bed.  
%
Since the gold-standard for this task is subjective and paramount, this project posits two questions about experts. 
%
Can relying on \textit{human-verified} datasets, specifically WikiData, set a higher standard for machine translation of question answering than is now possible?  
%
Additionally, how do you verify that a generative task with many possible options is providing a reasonable answer?

A challenge for modern data-hungry natural language processing
(\abr{nlp}) techniques is to replicate the impressive results for
standard English tasks and datasets to other languages.
%
Literally translating text into the target language is the most obvious solution.  
%
This can be the best option for tasks such as sentiment
analysis~\citep{araujo-16}, but for other tasks such as question
answering (\abr{qa}), literal translations might miss cultural nuance
if you directly translate questions from English to German to provide
additional training data.\footnote{Creating additional training data in any capacity---sequence to sequence generation, creating multi-lingual embeddings, transfer learning---requires an authoritative oracle in itself.}
%
While this might allow \abr{qa} systems to answer questions about
baseball and \entity{Tom Hanks} in German, it does not fulfill the promise of a
smart assistant answering a culturally-situated question about \entity{Oktoberfest}.

This alternative is called cultural \emph{adaptation}.
%
If you put a German sentence into a translation system, you might get
literal, correct translation like ``Mr. M\"uller grabbed a Berliner
from Dietsch at the Hauptbahnhof before jumping on the ICE''.
%
The cultural context of Germany is necessary to understand this example.

An extreme adaptation could render the sentence as
``Mr. Miller grabbed a Boston Cream from the Dunkin' Donuts in Grand
Central before jumping on the Acela'', elucidating that M\"uller
literally means ``Miller'', that Dietsch (like Dunkin' Donuts) is a
mid-range purveyor of baked goods, both Berliners and Boston Creams
are filled sweet pastries named after a city, and that the ICE is the
(slightly) ritzier inter-city train.
%
Humans translators use this type of adaptation frequently when it is appropriate to the translation.

Because adaptation is understudied, we leave the full translation task
to future work.
%
Instead, we focus on the task of cultural adaptation of entities: given an
entity in English, what is the corresponding entity in a target
language.
%
For example, the German \entity{Anthony Fauci}, the leading medical expert on coronavirus in the country, is \entity{Christian Drosten}.
%
Most Americans would be unfamiliar with this knowledge of German culture.
%
Hence, automatic adaptation could be used in natural language processing for machine translation and indirectly for generating new question answering datasets and education.   
%
Can machines reliably find these analogs with minimal supervision?
%
Answering this question requires working with experts to create a gold standard for this subjective analogy.  
\section{Was ist \textit{ George Washington}?}
\label{sec:motivation}

This section defines cultural adaptation and motivates it application for tasks
like creating culturally-centered training data for \abr{qa}.
%
\citet{vinay1995comparative} define adaptation as translation in which the relationship, and not the literal meaning, between the receiver and the content is recreated.

Work on analogy is close to our interest, but the standard analogy set-up lacks the cross-cultural and cross-lingual dimensions~\citep{turney2008uniform, gladkova-etal-2016-analogy}.
%
Additionally, recent methods for identifying entities or cross-lingual
translation could be repurposed for adaptation~\citep{duh2011machine,
	schnabel2015evaluation, kasai-etal-2019-low,
	arora-etal-2019-semi, kim-etal-2019-effective,
	hangya-fraser-2019-unsupervised}

Adaptation is most applicable when machine translation is combined
with other tasks.
%
Non-literal translation would be harmful for certain tasks such as the information retrieval of news stories.
%
In contrast, question answering is one domain where adaptation seems crucial.
%
There has been an explosion of English-language \abr{qa} data,
but not in other languages.  
%
Several approaches try to transfer English's bounty to other
languages.
%
\abr{mlqa} and \abr{xq}u\abr{ad} generate questions through machine
translation~\citep{lewis2019mlqa, 2019xquad}.
%
TyDi~\citep{tydiqa} gives users prompts from Wikipedia articles; other
datasets like \squad{} recapitulate the problematic distribution of
encyclopedias~\citep{reagle-11}.

Most of the entities asked about in major \abr{qa}
datasets---SQuAD, TriviaQA, Quizbowl---are American.
%
The coverage of the question remains the same across languages.

\textbf{ Given that we already have professionally-written questions,
	can we \textbf{adapt}, rather than literally generate, them to another culture and language?  }
\section{Adaptation from a Knowledge Base}
\label{sec:wikidata}

We first adapt entities using a knowledge base.
%
We use WikiData~\citep{vrandevcic2014wikidata}, a structured,
human-annotated representation of Wikipedia that is actively
developed.
%
This resource is well-suited to the task, particularly as features are
standardized both within and across languages.

Many knowledge bases explicitly encode the nationality of individuals,
places, and creative works.\footnote{Like with language, nationality
	is often correlated with culture, but is not synonymous.  Large
	countries contain multitudes, while some nationalities (e.g., Kurds)
	lack a \textit{de jure} nation but span many nations.  We elide this
	detail and focus on information often available in knowledge bases.}
%
Entities are represented in knowledge bases as discrete sparse
vectors, where most dimensions are unknown or not applicable (e.g., a building do not have a spouse).
%
For example, \entity{Angela Merkel} is a human (instance of), German
(country of citizenship), politician (occupation), Rotarian (member
of), Lutheran (religion), 1.65 meters tall (height), and has a PhD
(academic degree).
%
How would we find the ``most similar'' American adaptation to
\entity{Angela Merkel}?
%
Intuitively, we should find someone whose nationality is American.

Some issues immediately present themselves; contemporary entities will
have more non-zero entries than older entities.
%
Some characteristics are more important than others: matching unique
attributes like ``worked as journalist'' is more important than
matching ``is human''.
%
%Moreover, some attributes are discrete but others are continuous.

The items can be grouped by \textit{property} and by \textit{value}, the WikiData equivalent of intents and slots. 
%
\textit{Properties} in WikiData are the abstract intents: \entity{Merkel} has an ``occupation'',  a ``academic degree''.  
%
\textit{Values} are the slots: her ``occupation'' is ``politician'', her ``academic degree'' is  a ``doctorate''.
%
The former works for macro-entity classification since a building, a person and a song have
different properties.
Additionally,  more popular items have more properties.
%
The latter are useful \textit{within} a culture as \entity{Merkel}
will belong to a \textit{value} like the  \entity{Christian Democratic Union}, unlike an
American politician.

First, we bifurcate the WikiData into two sets: an American
set~$\mathcal{A}$ for items which contain the \textit{value} ``United
States of America'' and a German set~$\mathcal{D}$ for those with
German values.\footnote{While the geopolitical definition of
	American is straightforward, the German nation state is more
	nuanced~\citep{schulze-91}.  Following \citet{green-03}, we adopt
	members of the Zollverein or the German Confederation as ``German''
	as well as their prececessor and sucessor states.}
%
This is a liberal approximation, but it successfully excludes roughly
seven out of the eight million items in WikiData.
%
Then we explore the \textit{properties} and the \textit{values} from
the WikiData.
%
\textit{Properties} are limited and centrally organized.
%
\textit{Values} are more numerous and varying in quality.  
%
We select the highest frequency features.\footnote{Including a maximum
	and a minimum cap did not obviously generate better candidates than
	the most frequent items}
%
Values exist in all types of dimensions and the structure of WikiData is occasionally inconsistent.
%
For example, you will not find Goethe under any expected variations of Germany; he is only annotated under Saxe-Weimar-Eisenach. 
%
Including additional values does not lead to qualitatively better predictions with 20,000 values than with 1,000 values.  
%
We use \textit{properties} for our final results. 
%
%, but \textit{values} could prove useful for intra-culture adaptation.

The \textit{properties} are discrete and categorical;
\entity{Merkel} either has an ``occupation'' or she does not.
%
Each entity then has a sparse vector.
%
We calculate the similarity of the vectors with Faiss's~\citep{JDH17} L2 distance.  
%
Specifically, we search for each of the source German adaptation entities in the pre-selected 1,000,000 item American matrix.
%
Conversely we search for each of the American entities in the pre-selected 180,000 item German matrix.
%
This division is crucial as the most similar candidates are from the same cultural background.  

Formally we calculate the vector as:
\begin{equation}d' = \underset{a \in \mathcal{A}}{\mathrm{arg\,min}}  \| a - d \| ^2 \end{equation}
where $d'$ is the optimal German vector and $a \in \mathcal{A}$ are the items in the American matrix.

For both WikiData and the embedding-based approach, we select 100 candidates per item. 

So who is the American \entity{Angela Merkel}?
%
One possible answer is \entity{Woodrow Wilson}, a blue-eyed protestant who had a PhD, served as head of state, and
was also nominated for a Nobel Peace Prize.  
%
This answer may be unsatisfying as it was \entity{Barack Obama} who sat across from \entity{Merkel} for nearly a decade.
%
To capture these more nuanced similarities, we turn to large text corpora in Section~\ref{sec:embedding}.


While the classic \abr{nlp} vector
example~\citep{mikolov2013linguistic} isn't as magical as initially
claimed~\citep{rogers2017too}, it provides useful intuition.  We can use
the intuitions of the clich\'e:
%
\begin{equation}
\overrightarrow{\mbox{King}} - \overrightarrow{\mbox{Man}} + \overrightarrow{\mbox{Woman}} = \overrightarrow{\mbox{Queen}}
\end{equation}
to adapt between languages.  
%
% Just as we find WikiData vectors most similar apart from nationality, we solve:
We follow the word analogy approach of 3CosAdd~\citep{levy2014linguistic, koper2016improving} to adapt the source word by solving:
%
\begin{equation}
x - \overrightarrow{\mbox{American}} + \overrightarrow{\mbox{German}}
= \overrightarrow{\mbox{Merkel}}
\end{equation}
to find the closest entity, \underline{Obama}, to $x$.

%\jbgcomment{Given what's commented out, did you not do preprocessing to make entities single tokens?  If not, let's do that ASAP.}

Towards this end, we will need to create relevant embeddings.
%
First, we use Wikipedia dumps in the English and German language,
processed using Moses' preprocessing pipeline~\citep{koehn2007moses}.
%
However, by default, the dumps are separated as unigrams, whereas
Named Entities such as people are often phrases.
%
We follow \citet{mikolov2013distributed} and use co-occurrence
statistics to build bigrams and trigrams, limiting the vocabulary to
the 1M most frequent tokens.
%To address this limitation, we build bigrams and trigrams using cooccurence statistics, limiting the vocabulary to 
%
%
%On this phrase-based, and not exclusively unigram-based, data we are able to train cross-lingual embeddings using VecMap, which a leading method for English-German translation.  
We use word2vec~\citep{mikolov2013distributed}, rather than FastText~\citep{bojanowski2016enriching}, as we do not want orthography to influence the similarity of entities.
%
\entity{Merkel} in English and in German have quite different
neighbors, and we intend to keep it that way.
%

%\jbgcomment{what does ``orthography to have an effect'' mean?  Orthography doesn't have agency.}

However, the standard word2vec model assumes a single monolingual
embedding space.
%
To align the two monolingual spaces we use unsupervised
Vecmap~\citep{artetxe2018robust}, a leading tool for
cross-lingual word embeddings.
%
\jbgcomment{The above and below paragraphs seem inconsistent.  Vecmac
	is a rotation while below you're assuming a translation below.}
%
American$\rightarrow$German can be thought of as representing the source embedding in the American space and the target embedding in the German space.
%
Hence, the source (American) becomes \textit{x} in this equation, meaning that \textit{x-a+b} represents its adapted vector and the closest target words (German) based on cosine similarity its word adaptations.
%
\textit{a} and \textit{b} represent the American and German culture and are used as anchors for the adaptation. 
%
We average the vector of \entity{United States} in the English space and that of \entity{USA} in the German space for robustness. 
%
Similarly we average \entity{Germany} and \entity{Deutschland} for vector \textit{b}.
%
In standard analogy the \textit{a} and \textit{b} vectors are different for each test pair.  
%
In our case, the vectors are the same because the relation is identical for each \textit{x}-\textit{y} pair.  

Summarizing, we take the German (or American) embedding of the Named Entity, adapt it with 3CosAdd and look for the most similar words to the adapted embeddings in the American (or German) model.
%
In the case where the phrase is not found as an embedding, we back off to the last name of the named entity (e.g., \entity{Barack Obama} $\rightarrow$ \entity{Obama}).
\input{denis_proposal/sections/adaptation/50-embeddings}
\section{Evaluation by Experts }
\label{sec:eval}

The difficulty of the task merits skilled users.  
%
Since quality control is difficult for generation~\citep{peskov2019multi}, we need users who will answer the task accurately and without annotation artifacts.
%
We select five American citizens educated at American universities and five German citizens educated at German university.  
%
These human annotations serve as a gold standard against which we can compare our automated approaches.
%
To improve the user experience, we create a custom interface that:
\begin{enumerate}[noitemsep]
	\item describes the task and provides examples
	\item tracks the user inputting the annotation
	\item provides a brief summary from Wikipedia 
	\item pre-populates from an autocomplete box \textit{a la} answer selection in \citet{wallace-19}
\end{enumerate}
%
The annotation task requires roughly two hours for our users to complete. 
%
Our entities come from two sources: the top 500 most visited Wikipedia pages and the Veale NOC List~\citep{veale2016round}.
%
Wikipedia has a heavy skew towards pop culture; the top 500 pages had to be preemptively filtered to avoid being dependent on pop music and films.
%
The Veale NOC list is human-verified and contains a historically broader sweep of people.  
%
We conduct this exercise in both directions; while \entity{Berlin} is the German \entity{Washington, DC}, there is less consensus on what is the American \entity{Berlin}, as \entity{Berlin} is both the capital, a tech hub, and a film hub.  
%
We expect this dataset to show how
prototypical particular examples are within a culture.

\subsection{Question Adaptation}

The adaptation methodology allows us to solve a downstream task: machine translation of question answering.
%
We need a dataset of high-quality German questions, which does not exist at the moment.  
%
We will work with a German trivia company to create a dataset of high-quality German questions.  

Having gathered a German dataset we can automatically adapt the German questions into English.  
%
Conversely, we can adapt English questions~\citep{rajpurkar-16} into German.  
%
Experts will be used for evaluating both the content and the naturalness of the questions.  
%
If the quality of adapted named entities is insufficient for believable question generation, we can use a supervised learning approach to improve upon our proposed adaptation methodology.  

\subsection{Summary}

%jbgcomment{adaptation}

We propose entity adaptation as a task.
%
Word2vec embeddings and WikiData can be used to figuratively---not just literally---translate entities into a different culture.   
%
We are interested in knowing if both methods generate reasonable candidates. 
%
WikiData is largely human-verified and will test if crowd-sourced information is more similar to expert decision-making than automatic embeddings.  
%
Additionally, we will see how interpretable our predictions are.  
%
For our experiments we will create and release the first adaptation dataset for which citizens of the respective countries provide annotations for popular items from English and German Wikipedia, and a part of the Veale Non-Official Characterization list.
%












\begin{comment}
Modulation is important for translation being understandable in a target language.  
%
While studied in linguistics, this topic that has not been seriously studied by the Natural Language Processing community, despite having numerous downstream applications.   
%
One such downstream application, Question Answering, has been done through machine learning~\citep{iyyer-14b, rajpurkar-16, dunn-17, reddy-18, kwiatkowski-19, DBLP:journals/corr/abs-1904-04792}.  
%
But, most of the existing corpora are in English. 
%
 Machine translation has been used to extend this task to other languages~\citep{lewis2019mlqa, artetxe2019xquad}. 
 %
  However, literally translating a question referencing Named Entities does not necessarily make a question relevant through a cultural lens.  
  %
  We propose a method that can automatically find appropriate named entities across countries through \textit{human-interpretable} changes in WikiData features.  

Analyzing WikiData, we note a discrepancy in coverage of Germans and Americans.  Filtering by the country of citizenship we observe 295,820 Americans but only 46,081 Germans.  
%
This imbalance is significant but has enough Germans for our methodology.
%
As WikiData is a maintained resource, there is room for future additional coverage.  

This approach can be used for other countries.  For example, there are X Russian and Y Chinese nationals in WikiData.  

An approach like TyDi~\citep{tydiqa} generates natural questions in a language.  However, this is starting from scratch and relying on users to generate questions.  Given that we have known past quality questions from SQuAD, can we modulate, rather than literally generating them in an another language?  

We plug these named entities into X to improve downstream question answering.  We compare our sentences to the translated questions in XQuAD (or maybe MLQA).  Qualitative and quantitative analysis (question answering accuracy).   TBD Named entities are key.  How much we can improve question answering in target language?  

We need to decide between machine reading or answer selection. 

Cultural backgrounds have to be accounted for to achieve accurate and broadly-applicable machine translation.  Word embeddings from parallel corpora fail to adequately select candidates, since the named entities are translated literally.  

A popular method for creating word embeddings is MUSE.  Interestingly, named entities that occur in both German and English, occur next to different words.  

Nearest neighbors for Merkel in German are bundeskanzlerin, schäuble, and stoiber, which are all closely related in German politics.  In contrast, Merkel in English includes sarkozy, hollande, and erdoğan, who are not involved in German politics.  We see this pattern occur with an American politician: in English, Obama is close to Biden and McCain.  In German, Truman, Nixon, and Putin are thrown into the mix!  

Additionally, we explore multilingual alignment.  Visualization of word embeddings in a monolingual context works as expected.  But, visualizing English and German together leads to a clustering of words by language, rather than by subject matter.  German and English are isomorphic and aligning the embeddings of the two languages solves this issue.  
\end{comment}

