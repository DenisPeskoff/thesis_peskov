% intro
\section{Related Work}
\label{sec:lit}

Early computational deception work focuses on single
utterances~\citep{NewmanLyingwordsPredicting2003}, especially for
product reviews~\citep{OttEstimatingPrevalenceDeception2012}.
%%add Toma dating, warkenting forums
But deception is intrinsically a discursive phenomenon and thus the
context in which it appears is essential.
%
Our platform provides an opportunity to observe deception in the
context in which it arises: goal-oriented conversations around in-game
objectives.
%
Gathering data through an interactive game has a cheaper per-lie cost than hiring workers to
write deceptive statements~\citep{jurgens2014s}. 

%In the latter camp, we take into consideration factors that could
%affect data collection.  \dots

Other conversational datasets are mostly based on games that involve
deception including
Werewolf~\citep{GirleaPsycholinguisticFeaturesDeceptive2016}, Box of
Lies~\citep{SoldnerBoxLiesMultimodal2019}, and tailor-made games~\citep{HoEthicaldilemmaDeception2017}.
%
%conversation-level roles vs utterance-level annotations
However, these games assign individuals roles
that they maintain throughout the game (i.e., in a role that is
supposed to deceive or in a role that is deceived).
%
Thus, deception labels are coarse: an \emph{individual} always lies or
always tells the truth.
%
In contrast, our platform better captures a more multi-faceted reality
about human nature: everyone can lie or be truthful with everyone
else, and they use both strategically.
%
Hence, players must think about \textit{every} player lying at any
moment: ``given the evidence, do I think this person is lying to me
\emph{now}?''



Deception data with conversational labels is also available through
interviews \citep{Perez-RosasVerbalNonverbalClues2016}, some of which
allow for finer-grained deception
spans~\citep{LevitanLinguisticCuesDeception2018}.
%
Compared with game-sourced data, however, interviews provide shorter
conversational context (often only a single exchange with a few follow-ups)
and lack a strategic incentive---individuals lie because they are
instructed to do so, not to strategically accomplish a larger
goal.
%
In Diplomacy, users have an intrinsic motivation to lie; they have
entertainment-based and financial motivations to win the game.
%
This leads to higher-quality, creative lies.


Real-world examples of lying include perjury~\citep{louwerse2010linguistic},
calumny \citep{fornaciari2013automatic}, emails from malicious
hackers~\citep{Dhamija2006WhyPW}, and surreptitious user recordings.
%
But real-world data comes with real-world complications and privacy
concerns.
%
The artifice of Diplomacy allows us to gather pertinent language data
with minimal risk and to access both sides of deception: intention and
perception.
%
Other avenues for less secure research include analyzing dating
profiles for accuracy in self-presentation~\citep{toma2012lies} and classifying deceptive online spam~\citep{ott2011finding}.




