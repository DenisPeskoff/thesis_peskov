Expert-created datasets can create datasets of a quality unachievable with generalists.  
%
This has a twofold benefit.
%
First, the accuracy, rather than the size, of the data allows the dataset to withstand the test of time.\footnote{The Penn Treebank~\citep{marcus1993building2}, which used graduate students in linguistics and spanned three years in the early 1990s, remains a staple of Computational Linguistics curriculum today}.
%
Second, tasks that require long-term commitment and complexity become possible.  
%
This result justifies the large investment of time, relationship-building, and money.  

As a contrast to the earlier crowd-sourced work, we create a deception dataset using only experts.
%
Participants---that are engaged in the task and  are appropriately compensated---both \textit{generate} and \textit{annotate} data in the span of a game that usually lasts over a month. 
%  
The resulting product is a gold standard of conversational \nlp{} data in terms of quality of language, diversity, and naturalness.\footnote{Denis Peskov, Benny Chang, Ahmed Elgohary, Joe Barrow, Cristian Danescu-Niculescu-Mizil, and Jordan Boyd-Graber. 2020. It Takes Two to Lie: One to Lie and One to Listen. In Proceedings of The Association for Computational Linguistics.\\
Peskov was responsible for designing the task, gathering the participants, running the games, building half the models, part of the data analysis, the visualizations, and the paper writing.}  
%
The conversations and annotations thereof would not be possible without experts from the community.

\section{Where Does One Find Long-Term Deception?}
\label{sec:intro}


A functioning society is impossible without trust.
%
In online text interactions, users are typically
trusting~\citep{shneiderman2000designing}, but this trust can be
betrayed through false identities on dating sites~\citep{toma2012lies},
spearphishing attacks~\citep{Dhamija2006WhyPW},
sockpuppetry~\citep{kumar_army_2017} and, more broadly, disinformation
campaigns~\citep{kumar_false_2018}.
%
Beyond such one-off antisocial acts directed at strangers, deception
can also occur in sustained relationships, where it can be
strategically combined with truthfulness to advance a long-term
objective~\citep{ Cornwell2001LoveOT, kaplar2004enigma}.

% Computational methods to understand and mitigate such anti-social behaviors have relied on the availability comprehensive datasets with gold-label annotations.
%%pre DNM
% A functioning society is impossible without trust.
% %
% In online text interactions, users are typically
% trusting~\citep{shneiderman2000designing}, but this trust can be
% betrayed through false identities on dating sites~\citep{toma2012lies},
% spearphishing attacks~\citep{Dhamija2006WhyPW}, and malicious
% rumors~\citep{Qazvinian2011RumorHI}.
% %
% Other violations of the online social compact such as spam
% e-mail have been mitigated through machine learning~\citep{graham-03}:
% find the problem and hide it from view.
% %
% However, this approach has not been applied to text deception because
% we lack large enough gold-standard deception datasets for training
% machine learning algorithms.
%\jbgcomment{Address Cristian's point that  this dataset is unique because it has sustained interactions, not just one-offs}

We introduce a dataset to study the strategic use of
deception in long-lasting relationships.
%
To collect reliable ground truth in this complex scenario, we design
an interface for players to naturally generate and annotate
conversational data while playing a negotiation-based game called
Diplomacy.
%
These annotations are done in \textit{real-time} as the players send
and receive messages.
%
While this game setup might not directly translate to real-world
situations, it enables computational frameworks for studying deception
in a complex social context while avoiding privacy issues.


%%% pre DNM
% This work rectifies this lacuna through a large dataset of
% creative messages annotated with deception---not just one-off messages but sustained interactions in a complex relationship.
% %
% We design an interface that allows users to naturally generate and
% annotate conversational data while playing the game called Diplomacy.
% %
% These annotations are done in \textit{real-time} on sent and received
% messages.  
% %
% Using a game allows us to avoid privacy issues in a research topic fraught with them.  
% %


%\jbgcomment{Troy is a good reference, but it isn't a text-based
  %interaction, so we should probably avoid it.  Reworked new
 % paragraph, but needs citeps.}

%DONE \jbgcomment{ This is good, but should probably be in related
%work with appropriate citations: court transcripts of perjury, emails
%from malicious hackers, and surreptitious user recordings.  But
%real-world data comes with real-world complications and privacy
%concerns. Therefore, we use an artificial construct that allows us to
%gather pertinent language data without any risk.}

%moved\jbgcomment{ Again, this is good, but probably should be later
%in a section about our formal definition:}

\begin{table}[t]
	\centering
	\small
	\begin{tabular*}{\linewidth}{p{10cm}p{2cm}p{2cm}}
		{\bf Message} & {\bf Sender's intention} & {\bf Receiver's percep.} \\
		\hline
		\noalign{\vskip 2mm} 
		\parbox{10 cm}{If I were lying to you, I'd smile and
                  say ``that sounds great.''  I'm honest with you
                  because I sincerely thought of us as partners.}
                & Lie & Truth \\
		\noalign{\vskip 2mm} 
		\rowcolor{gray!25}
		\parbox{10 cm}{You agreed to warn me of unexpected
                  moves, then didn't \dots You've revealed things to
                  England without my permission, and then made up a
                  story about it after the fact!}  & Truth & Truth \\
		\noalign{\vskip 2mm}
		\parbox{10 cm}{\dots I have a reputation in this hobby
                  for being sincere.  Not being duplicitous.  It has
                  always served me well. \dots If you don't want to
                  work with me, then I can understand that \dots} &
                Lie & Truth \\
		\noalign{\vskip 2mm} 
		\rowcolor{gray!25}
		\multicolumn{3}{c}{\textit{(Germany attacks Italy)}} \\ 
		\noalign{\vskip 2mm} 
		\parbox{10 cm}{Well this game just got less fun} & Truth& Truth \\
		\noalign{\vskip 2mm} 
		\rowcolor{gray!25}
		\parbox{10 cm}{For you, maybe} & Truth & Truth \\
	\end{tabular*}
	\caption{An annotated conversation between \player{Italy} (white) and \player{Germany} (gray) at a moment when their relationship breaks down.  Each message is annotated by the sender (and
          receiver) with its intended or perceived truthfulness; \player{Italy} is lying about \dots lying.}
	\label{tab:dialogexample}
\end{table}

After providing background on the game of Diplomacy and our intended deception annotations (Section~\ref{sec:diplo}),
%
we discuss our study (Section~\ref{sec:study}).
%
To probe the value of the resulting dataset, we develop lie prediction
models (Section~\ref{sec:models}) and analyze their results
(Section~\ref{sec:analysis}).
% Finally, we discuss this dataset within the deception literature
%(Section~\ref{sec:lit}) and discuss the relevance of studying
%deception in the context of long-lasting relationships
%(Section~\ref{sec:contribution}).


