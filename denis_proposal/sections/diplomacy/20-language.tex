\section{Diplomacy}
\label{sec:diplo}


The Diplomacy board game places a player in the role of one of seven
European powers on the eve of World War I.
%
The goal is to conquer a simplified map of Europe by ordering armies
in the field against rivals.
%
Victory points determine the success of a player and allow them to build
additional armies; the player who can gain and maintain the highest
number of points wins.\footnote{In the parlance of Diplomacy games,
  points are ``supply centers'' in specific territories (e.g.,
  London).  Having more supply centers allows a player to build more
  armies and win the game by capturing more than half of the 34 supply
  centers on the board.}
%
The mechanics of the game are simple and deterministic: armies,
represented as figures on a given territory, can only move to adjacent
spots and the side with the most armies always wins in a disputed
move.
%
The game movements become publicly available to all players after the
end of a turn. \dpcomment{Chris M suggested adding sentence
  contrasting this with other information games}
%

Because the game is deterministic and 
everyone begins with an equal amount of armies, a player cannot win
the game without forming alliances with other players---hence the name
of the game: Diplomacy. \dpcomment{Chris M suggestion to inverse:
  Everyone begins with the same amount of armies, and because conflict
  is decided by superior force, a player cannot win the game without
  forming alliances}
%
Conquering neighboring territories depends on support from another
player's armies.
%
After an alliance has outlived its usefulness, a player often
dramatically breaks it to take advantage of their erstwhile ally's
vulnerability.
%
Table~\ref{tab:dialogexample} shows the end of one such relationship.
%
As in real life, to succeed a betrayal must be a surprise to the
victim.
%
Thus, players pride themselves on being able to lie \emph{and} detect lies.
%
Our study uses their skill and passion to build a dataset of deception
created by battle-hardened diplomats.
%
Senders annotate whether each message they write is an \alie{} and
recipients annotate whether each message received is a \slie{}.
%
Further details on the annotation process are in
Section~\ref{sec:interface}.
%
%The annotation reveals the sender's internal state, at the time the message is sent.  


\subsection{A game walk-through} 
\label{sec:walkthrough}

Figure~\ref{fig:full_game} shows the raw counts of one game
in our dataset.
%
But numbers do not tell the whole story.
%
We analyze this case study using rhetorical
tactics~\citep{cialdini2004social}, which \citet{oliveira2017dissecting}
use to dissect spear phishing e-mails and \citet{anand-11} apply to
persuasive blogs.
%
Mentions of tactics are in italic (e.g., \textit{authority}.
%; context for quotes in Appendix, Table~\ref{tab:persuasion}. 
%
For the rest of the paper, we will refer to players via the name of
their assigned \player{country}.
%

\begin{figure}[t]
	\centering
	\includegraphics[width=.75\linewidth, height = 13cm]{\autofig{game_lies.pdf}}
	\caption{Counts from one game featuring an \player{Italy}
          (green) adept at lying but who does not fall for others'
          lies.  The player's successful lies allow them to gain an
          advantage in points over the duration of the game.  In 1906,
          \player{Italy} lies to \player{England} before breaking
          their relationship.  In 1907, \player{Italy} lies to
          everybody else about wanting to agree to a draw, leading to
          the large spike in successful lies.}
	\label{fig:full_game}
\end{figure}

Through two lie-intense strategies---convincing \player{England} to betray
\player{Germany} and convincing all remaining countries to agree to a
draw---\player{Italy} gains control of the board.
%
\player{Italy}'s first deception is a plan with \player{Austria} to
dismantle \player{Turkey}.
%
\player{Turkey} believes \player{Italy}'s initial assurance of
non-aggression in 1901.
%
\player{Italy} begins by excusing his initial silence due to a rough day at
work, evoking empathy and \textit{likability}.
%
While they do not fall for subsequent lies, \player{Turkey}'s initial
gullibility cements Italy's first-strike advantage.
%
Meanwhile, \player{Italy} proposes a long-term alliance with \player{England}  against
\player{France}, packaging several small truths with a big lie.
%
The strategy succeeds, eliminating \player{Italy}'s greatest threat.

Local threats eliminated, \player{Italy} turns to rivals on the other
end of the map.
%
\player{Italy} persuades \player{England} to double-cross its
long-time ally \player{Germany} in a moment of \textit{scarcity}: if
you do not act now, there will be nowhere to expand.
%\jbgcomment{This didn't explain how this was scarcity, I added my best guess.}
%
\player{England} accepts help from ascendant Italy, expecting
\textit{reciprocity}.
%
However, \player{Italy} aggressively and successfully moves against
\player{England}.
%
The last year features a meta-game deception.
%
After \player{Italy} becomes too powerful to contain, the remaining four
players team up.
%
Ingeniously, \player{Italy} feigns acquiescence to a five-way draw,
individually lying to each player and establishing \textit{authority}
while brokering the deal.
%
Despite \player{Italy's} record of deception, the other players believe the
proposal (annotating received messages from \player{Italy} as truthful) and
expect a 1907 endgame, the year with the most lies.
%
\player{Italy}  goes on the offensive and knocks out \player{Austria}.
%
%\player{Italy}'s summary of the game in their own words is in the Appendix, Table~\ref{tab:userfeedback}.
%\jbgcomment{Give more explicit forward reference so that readers don't have to hunt.}

%\jbgcomment{Again, focus on who's lying}

%Cialdini seems to best citep and it's used by cybseceurity \jbgcomment{Link the examples to the literature.  Presumably these things have names, find out what those names are, and then citep a relevant paper.  Talk about how that's relevant for things like Phishing and social engineering.}

Each game has relationships that are forged and then riven.
%
In another game, an honest attempt by a strong \player{Austria} to woo an
ascendant \player{Germany} backfires, knocking \player{Austria} from the game.
%
\player{Germany} builds trust with \player{Austria} through a believed
fictional experience as a Boy Scout in Maine (\textit{likability}).
%
In a third game, two consecutive unfulfilled promises by an ambitious
\player{Russia} leads to a quick demise, as their subsequent excuses
and apologies are perceived as lies (failed \textit{consistency}).
%
In another game, \player{England}, \player{France}, and
\player{Russia} simultaneously attack \player{Germany} after offering
duplicitous assurances.
%
Game outcomes vary despite the identical, balanced starting board, as
different players use unique strategies to persuade, and occasionally
deceive, their opponents.
  
\subsection{Defining a lie}

%added citeps and additional definitions \jbgcomment{This needs to be fleshed out; need more citations.  E.g., citep things that look at fake profiles, reviews, etc. citep some of the alternate definitions of lies, talk about why we
  %don't use them.  }

Statements can be incorrect for a host of reasons: ignorance,
misunderstanding, omission, exaggeration.
%
\citep{gokhman2012search} highlight the difficulty of finding willful,
honest, and skilled deception outside of short-term, artificial
contexts~\citep{depaulo2003cues}.
%
Crowdsourced and automatic datasets rely on simple
negations~\citep{perez2017automatic} or completely implausible claims
(e.g., ``Tipper Gore was created in 1048'' from \citep{fever-18}).
%
While lawyers in depositions and users of dating sites will not
willingly admit to their lies, the players of online games are more
willing to revel in their deception.

We must first define what we mean by deception.
%
Lying is a mischaracterization; it's thus no surprise that a
definition may be divisive or the subject of academic
debate~\citep{gettier-63}.
%
We provide this definition to our users: ``Typically, when [someone]
lies [they] say what [they] know to be false in an attempt to deceive
the listener''~\citep{siegler1966lying}.
%
An orthodox definition requires the speaker to utter an explicit
falsehood~\citep{sep-lying-definition}; skilled liars can deceive with
a patina of veracity.
%
A similar definition is required for prosecution of perjury, leading
to a paucity of convictions~\citep{bogner-74}.
%
Indeed, when we ask participants what a lie looks like, they mention
evasiveness, shorter messages, over-qualification, and creating false
hypothetical scenarios~\citep{depaulo2003cues}.

\subsection{Annotating truthfulness}

\begin{figure}[t]
	\footnotesize
	\centering
	\includegraphics[width = \linewidth]{\figfile{Interface_v3.pdf}}
	\caption{Every time they send a message, players say whether
          the message is truthful or intended to deceive.  The receiver then
          labels whether incoming messages are a lie or not.  Here
          \player{Italy} indicates they believe a message from
          \player{England} is truthful but that their reply is not. }
        %\jbgcomment{Too small.  If this is to be included, make font bigger or zoom in to important regions.}
	\label{fig:interface}
\end{figure}


%\begin{figure*}[t]
%	\centering
%	\includegraphics[width=\linewidth]{\figfile{convo_example_v1.png}}
%	\caption{An example conversation from our messaging interface.  Users write messages and annotate them with thumbs up or thumbs down.  Germany has accused Italy of being an untrustworthy ally.}
%\end{figure*}


% don't think CHI comment relevant anymore \jbgcomment{\dpcomment{need help.  it's technically reliable, it looks like the game, and it's visually friendly.  How do I write this and how do I citep this a la CHI?}You can't just say ``easiest to use''.  Talk about what
	%made them easy to use with the language of CHI (affordances, idioms,
	%transfer effects, etc.).}

%The board and move placement is provided as Figure~\ref{fig:board}  
%Discord has been used for educational research in the past~\citep{lacher2018using}.

%done\jbgcomment{Distinction with prior work needs to be clearer.  That was	post-hoc; this is in the loop.  That focused on betrayal inferred from Talk about how that might affect people's patterns of lying, etc.  (Also, add citeps.)}

%\dpcomment{should all of below be a paragraph in the lit review instead}
%\jbgcomment{It doesn't have to be relegated to related work, but should be here like this.  Introduce this as you use the features.}

Previous work on the language of Diplomacy~\citep{niculaelinguistic}
lacks access to players' internal state and was limited to
\textit{post-hoc} analysis.
%
We improve on this by designing our own interface that gathers
players' intentions and perceptions in real-time
(Section~\ref{sec:interface}).
%
As with other highly subjective phenomena like
sarcasm~\citep{gonzalez-ibanez_identifying_2011,bamman_contextualized_2015},
sentiment~\citep{pang2008opinion} and framing~\citep{greene2009more},
the intention to deceive is reflective on someone's internal state.
%
Having individuals provide their own labels for their internal state
is essential as third party annotators could not accurately access
it~\citep{chang_dont_2020-1}.

Most importantly, our gracious players have allowed this language data
to be released in accordance with \abr{irb} authorized anonymization,
encouraging further work on the strategic use of deception in
long-lasting relations.\footnote{Data available at
  \url{http://go.umd.edu/diplomacy_data} and as part of ConvoKit
  \url{http://convokit.cornell.edu}.}

\section{Broader Applicability}

This differs from previous work that does not follow the expert-generated paradigm.  
%
The most prominent past work on Diplomacy in the \abr{nlp} community, \citep{niculaelinguistic}, did not collect their data and thus could not release it to the public.  
%
This hampers follow-up applications of the research; a believable Diplomacy-playing (and speaking) bot cannot be trained if the raw language data is redacted and shuffled.  
%
We believe this work can set a paradigm for work outside of Diplomacy, and even \abr{nlp}; the interface created for this project, as well as the pre and post-game user surveys can be modifying for any conversational task.  
%
Most importantly, building a relationship with data generators elevates the standard of the data and guarantees its liberal distribution.  
%
Further work is necessary in codifying data standards---Show Your Data, not only your Work\footnote{\citep{dodge2019show}}.
