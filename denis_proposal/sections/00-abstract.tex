%Abstract Page

\hbox{\ }

\renewcommand{\baselinestretch}{1}
\small \normalsize

\begin{center}
    \large{{ABSTRACT}}
    
    \vspace{3em}
    
\end{center}
\hspace{-.15in}
\begin{tabular}{ll}
    Title of proposal: & {\large  Gathering Language Data Using Experts}     \\
    \                                                                      \\
                       & {\large  Denis Peskov, 2020}                   \\
    %&                           {\large Doctor of Philosophy, 2004} \\
    \                                                                      \\
    Dissertation directed by:
                       & {\large  Professor Jordan Boyd-Graber}            \\
                       & {\large  Department of Computer Science}          \\
                       & {\large  College of Information Studies}          \\
                       & {\large  Language Science Center}                 \\
                       & {\large  Institute for Advanced Computer Studies} \\
\end{tabular}

\vspace{3em}

\renewcommand{\baselinestretch}{2}
\large \normalsize


Natural Language Processing needs substantial data to make robust predictions. 
%
Automatic methods, unspecialized crowds, and domain experts can be used to collect this prerequisite data.
%
We curate large-scale conversational and question answering \abr{nlp} datasets using these various methods. 


A low-cost, high-output approach to data creation is \textit{automation}.  
%
We explore this approach by creating and analyzing a large-scale audio question answering dataset through text-to-speech technology.  
%
%However, in Quizbowl questions are read at an unusually fast pace and involve highly technical and multi-cultural words causing a disparity between automation and reality.  
%
We conclude that the cost-savings and scalability of automation come at the cost of data quality and naturalness.  

Human input can provide this degree of naturalness, but is limited in scale. 
%
Hence, large-scale data collection is frequently done through \textit{crowd-sourcing}.
%
A question-rewriting task, in which a long information-gathering conversation is used as source material for many stand-alone questions, shows the limitation of using this methodology for \textit{generating} data.
%; however, these have limitations as noted in two question answering projects: question-rewriting and audio question answering.     
%
%Generating data through crowd-sourcing poses a serious quality control challenge, since 
Standard inter-annotator agreement metrics, while useful for \textit{annotation}, cannot easily evaluate \textit{generated} data, causing a serious quality control issue.  
%
This problem is observed while formalizing a question-rewriting task; certain users provide low-quality rewrites---removing words from the question, copy and pasting the answer into the question---if left unsupervised.  
%
We automatically prevent unsatisfactory submissions through an interface, but the quality control process still requires manually reviewing over 5,000 questions.  

Certain users provide more reliable annotations and generations than others, which can be used to improve the quality control process.  
%
%We mitigate the quality control issues identified in crowd-sourcing and automation through exploring
\textit{Hybrid} solutions pair potentially unreliable and unverified users in the crowd with experts.  
%
As an example, Amazon customer service agents are used for curation and annotation of goal-oriented 81,000 conversations across six domains.  
%
By grounding the conversation with a reliable conversationalist---the Amazon agent---we create untemplated conversations and reliably identify low-quality conversations.    
%
The sentences generated from crowd workers are less natural and diverse than those from experts.   

Therefore, we posit that exclusively using domain \textit{experts} for data generation can create novel and reliable \abr{nlp} datasets. 
%
In a study on the game of Diplomacy, which investigates the language of trust and deception, Diplomacy community members generate a corpus of 17,000 messages that are self-annotated while playing a game. 
%
The language is varied in length, tone, vocabulary, punctuation, and even emojis.
%
Additionally, we create a real-time self-annotation system that annotates deception in a manner not possible through  crowd-sourced or automatic methods.  
%Annotation based on a user's perception can change ex-post-facto; a duped user may in retrospect conclude that they had expected a lie when reality suggests otherwise.   
 %
%Hence, real-time annotation by the original user is integral to this task.  
 %
We propose future work that leverages experts to create a new machine translation task: \textit{cultural adaptation}.  
 %
%Identifying relevant communities for a specific NLP task, and providing a service to them can set new standards for NLP corpora.  
 
