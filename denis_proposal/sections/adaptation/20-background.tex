\section{Using Cultural Experts for Translation}
\label{sec:propmod}
Chapter~\ref{ch:hybrid} proposes a method to evaluate machine translation models and in turn data.   
%
%MOVED TO earlier chapter %If we can establish that neural models are shallow in their understanding of a task, we should be able to establish that current auto-generated or crowd-sourced datasets are insufficient in quality.
%
The major limitation of the work is that the crowd-sourced workers generate less authentic sentences than the experts.  
%
How then can we generate data at scale, but with a level of reliability? 

Adaptation is such a task that combines our past work in Question Answering, with our proposed work in Machine Translation and is a good, difficult test-bed.  
%
Since the gold-standard for this task is subjective and paramount, this project posits two questions about experts. 
%
Can relying on \textit{human-verified} datasets, specifically WikiData, set a higher standard for machine translation of question answering than is now possible?  
%
Additionally, how do you verify that a generative task with many possible options is providing a reasonable answer?

A challenge for modern data-hungry natural language processing
(\abr{nlp}) techniques is to replicate the impressive results for
standard English tasks and datasets to other languages.
%
Literally translating text into the target language is the most obvious solution.  
%
This can be the best option for tasks such as sentiment
analysis~\citep{araujo-16}, but for other tasks such as question
answering (\abr{qa}), literal translations might miss cultural nuance
if you directly translate questions from English to German to provide
additional training data.\footnote{Creating additional training data in any capacity---sequence to sequence generation, creating multi-lingual embeddings, transfer learning---requires an authoritative oracle in itself.}
%
While this might allow \abr{qa} systems to answer questions about
baseball and \entity{Tom Hanks} in German, it does not fulfill the promise of a
smart assistant answering a culturally-situated question about \entity{Oktoberfest}.

This alternative is called cultural \emph{adaptation}.
%
If you put a German sentence into a translation system, you might get
literal, correct translation like ``Mr. M\"uller grabbed a Berliner
from Dietsch at the Hauptbahnhof before jumping on the ICE''.
%
The cultural context of Germany is necessary to understand this example.

An extreme adaptation could render the sentence as
``Mr. Miller grabbed a Boston Cream from the Dunkin' Donuts in Grand
Central before jumping on the Acela'', elucidating that M\"uller
literally means ``Miller'', that Dietsch (like Dunkin' Donuts) is a
mid-range purveyor of baked goods, both Berliners and Boston Creams
are filled sweet pastries named after a city, and that the ICE is the
(slightly) ritzier inter-city train.
%
Humans translators use this type of adaptation frequently when it is appropriate to the translation.

Because adaptation is understudied, we leave the full translation task
to future work.
%
Instead, we focus on the task of cultural adaptation of entities: given an
entity in English, what is the corresponding entity in a target
language.
%
For example, the German \entity{Anthony Fauci}, the leading medical expert on coronavirus in the country, is \entity{Christian Drosten}.
%
Most Americans would be unfamiliar with this knowledge of German culture.
%
Hence, automatic adaptation could be used in natural language processing for machine translation and indirectly for generating new question answering datasets and education.   
%
Can machines reliably find these analogs with minimal supervision?
%
Answering this question requires working with experts to create a gold standard for this subjective analogy.  