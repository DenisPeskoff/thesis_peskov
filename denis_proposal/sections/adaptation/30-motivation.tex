\section{Was ist \textit{ George Washington}?}
\label{sec:motivation}

This section defines cultural adaptation and motivates it application for tasks
like creating culturally-centered training data for \abr{qa}.
%
\citet{vinay1995comparative} define adaptation as translation in which the relationship, and not the literal meaning, between the receiver and the content is recreated.

Work on analogy is close to our interest, but the standard analogy set-up lacks the cross-cultural and cross-lingual dimensions~\citep{turney2008uniform, gladkova-etal-2016-analogy}.
%
Additionally, recent methods for identifying entities or cross-lingual
translation could be repurposed for adaptation~\citep{duh2011machine,
	schnabel2015evaluation, kasai-etal-2019-low,
	arora-etal-2019-semi, kim-etal-2019-effective,
	hangya-fraser-2019-unsupervised}

Adaptation is most applicable when machine translation is combined
with other tasks.
%
Non-literal translation would be harmful for certain tasks such as the information retrieval of news stories.
%
In contrast, question answering is one domain where adaptation seems crucial.
%
There has been an explosion of English-language \abr{qa} data,
but not in other languages.  
%
Several approaches try to transfer English's bounty to other
languages.
%
\abr{mlqa} and \abr{xq}u\abr{ad} generate questions through machine
translation~\citep{lewis2019mlqa, 2019xquad}.
%
TyDi~\citep{tydiqa} gives users prompts from Wikipedia articles; other
datasets like \squad{} recapitulate the problematic distribution of
encyclopedias~\citep{reagle-11}.

Most of the entities asked about in major \abr{qa}
datasets---SQuAD, TriviaQA, Quizbowl---are American.
%
The coverage of the question remains the same across languages.

\textbf{ Given that we already have professionally-written questions,
	can we \textbf{adapt}, rather than literally generate, them to another culture and language?  }